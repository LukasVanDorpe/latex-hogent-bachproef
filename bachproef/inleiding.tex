%%=============================================================================
%% Inleiding
%%=============================================================================

\chapter{\IfLanguageName{dutch}{Inleiding}{Introduction}}%
\label{ch:inleiding}

In de evenementenbranche wordt er steeds meer gebruikgemaakt van vernieuwde technologieën om het gebruiksgemak van de bezoekers te vergroten en de veiligheid te garanderen. Denk aan contactloze betalingen aan de hand van NFC chips en QR-codes om toegangskaartjes te scannen. Het veilig opbergen van persoonlijke spullen kan vanaf nu ook gemoderniseerd worden. Het bedrijf Lockit Rentals heeft hiervoor in 2022 QR-code lockers ontworpen en geïntroduceerd. Met die unieke QR-code kunnen klanten hun persoonlijke spullen veilig opbergen en de locker onbeperkt openen voor een vaste opgelegde periode. De problemen ontstaan als de toegangscode die voorgesteld wordt als een QR-code niet meer uitgelezen kan worden of wanneer de klant zijn/haar QR-code verliest. 

\newpage

\section{\IfLanguageName{dutch}{Probleemstelling}{Problem Statement}}%
\label{sec:probleemstelling}

Bij het digitaliseren van lockers op evenementen kan er een uitdaging ontstaan doordat de festivalgangers zelf verantwoordelijk zijn voor de inhoud van het kluisje, evenals hun gekochte toegangscode. Deze toegangscode, die als QR-code wordt voorgesteld kan bewaard worden op allerlei manieren. Enkele mogelijkheden zijn: bewaren als bijlage, bewaren op de mobiele applicatie van Lockit Rentals of wanneer als effectieve QR-code voorgesteld op papier. Uiteraard is er geen zekerheid dat de festivalganger zijn code in zijn bezit heeft tot op het einde van een evenement. De huurder zijn telefoon kan uitgevallen zijn waardoor dat de persoon in kwestie de QR-code niet meer in bezit heeft of de festivalganger speelt de afdrukte versie van de QR-code kwijt. Voor dit probleem te voorkomen zal gedurende de uitbating logistiek personeel aanwezig moeten zijn. Deze kunnen aan de hand van administratierechten een duplicaat QR-code genereren om zo de locker te openen. Lockit Rentals wil dit probleem doeltreffend oplossen zodat hun bedrijfsactiviteiten verbeteren. \newline 

Een extra problematiek doet zich voor als de QR-code scanners niet optimaal functioneren. Deze problemen kunnen ontstaan door onveranderbare factoren zoals hevige zon of felle regen. Eveneens door QR-code scanners die foutief zijn afgesteld of zijn aangetast tijdens het transport of verplaatsing van de lockers. Ook hebben mensen die niet op de hoogte zijn van het systeem extra hulp nodig bij het scannen. Door het willen afschaffen van de logistieke medewerkers verliezen ze hier de mogelijkheid om informatieve en technische hulp aan te bieden ten behoeve van de klanten. Na festivalzomer 2022 is gebleken uit de observaties van het bedrijf dat de technologieën niet genoeg uitgewerkt zijn om de units onbemand uit te baten.

\newpage
\section{\IfLanguageName{dutch}{Onderzoeksvraag}{Research question}}
\label{sec:onderzoeksvraag}

\subsection{Hoofdonderzoeksvragen}}
Zoals al aangegeven in sectie \ref{sec:probleemstelling} zal dit onderzoek vooral focussen op het vinden van betere en snellere oplossingen voor de problemen waarmee Lockit Rentals vandaag de dag mee kampt. Deze te onderzoeken feiten kan worden samengevoegd in één hoofdonderzoeksvraag.

\begin{itemize}
    \item Welke technologieën kan het bedrijf Lockit Rentals toepassen om hun bedrijfsactiviteiten bij het verhuren van lockers te optimaliseren?
\end{itemize}

\subsection{Deelonderzoeksvragen}}

Om de hoofdonderzoeksvraag verder op te splitsen kunnen we ook nog enkele deelvragen stellen.

\begin{itemize}
    \item Welke gebruikte technieken kunnen we optimaliseren om zo de doelstelling te behalen?
    \item Hoe zijn deze hedendaagse technieken gebruikt binnen de bedrijfsactiviteiten?
\end{itemize}

%TODO: geef extra vragen

Op deze vragen wordt doorheen het onderzoek een duidelijk antwoord aangeboden. De volledige antwoorden op de opgestelde onderzoeksvragen zijn te vinden in de conclusie \ref{ch:conclusie}.


\section{\IfLanguageName{dutch}{Onderzoeksdoelstelling}{Research objective}}%
\label{sec:onderzoeksdoelstelling}
\begin{comment}
    Wat is het beoogde resultaat van je bachelorproef? Wat zijn de criteria voor succes? Beschrijf die zo concreet mogelijk. Gaat het bv.\ om een proof-of-concept, een prototype, een verslag met aanbevelingen, een vergelijkende studie, enz.
\end{comment}


\section{\IfLanguageName{dutch}{Opzet van deze bachelorproef}{Structure of this bachelor thesis}}%
\label{sec:opzet-bachelorproef}

% Het is gebruikelijk aan het einde van de inleiding een overzicht te
% geven van de opbouw van de rest van de tekst. Deze sectie bevat al een aanzet
% die je kan aanvullen/aanpassen in functie van je eigen tekst.
\begin{comment}
    De rest van deze bachelorproef is als volgt opgebouwd:
    
    In Hoofdstuk~\ref{ch:stand-van-zaken} wordt een overzicht gegeven van de stand van zaken binnen het onderzoeksdomein, op basis van een literatuurstudie.
    
    In Hoofdstuk~\ref{ch:methodologie} wordt de methodologie toegelicht en worden de gebruikte onderzoekstechnieken besproken om een antwoord te kunnen formuleren op de onderzoeksvragen.
    
    % TODO: Vul hier aan voor je eigen hoofstukken, één of twee zinnen per hoofdstuk
    
    In Hoofdstuk~\ref{ch:conclusie}, tenslotte, wordt de conclusie gegeven en een antwoord geformuleerd op de onderzoeksvragen. Daarbij wordt ook een aanzet gegeven voor toekomstig onderzoek binnen dit domein.
\end{comment}
