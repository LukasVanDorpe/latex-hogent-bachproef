%%=============================================================================
%% Methodologie
%%=============================================================================

\chapter{\IfLanguageName{dutch}{Methodologie}{Methodology}}%
\label{ch:methodologie}

%% TODO: Hoe ben je te werk gegaan? Verdeel je onderzoek in grote fasen, en
%% licht in elke fase toe welke stappen je gevolgd hebt. Verantwoord waarom je
%% op deze manier te werk gegaan bent. Je moet kunnen aantonen dat je de best
%% mogelijke manier toegepast hebt om een antwoord te vinden op de
%% onderzoeksvraag.

Zoals bij de start van elk onderzoek is het verwerven van allerhande informatie over het onderwerp het belangrijkste aspect. Deze informatie omvat de implementatie van de locker verhuur en wordt gecapteerd in de literatuurstudie die te vinden is in hoofdstuk \ref{ch:stand-van-zaken}. Er wordt een duidelijk onderscheid gemaakt tussen de werking van QR-codes en het ontwikkelen van een chatbot. 

Als de informatie verwerkt is, kunnen beschikbare opties voor mogelijke oplossingen bekeken en aangeboden worden. Om deze fase zo goed mogelijk te vervolledigen moeten de requirements verzameld en geanalyseerd worden. De requirements zorgen voor een duidelijke afbakening. Uit de voorop gestelde eisen zal volgen welke technologieën geschikt zijn om de probleemstelling op te lossen. Nadien volgt een uitwerking met bijhorende experimenten die terug te vinden zijn in het hoofdstuk van de praktische Proof-Of-Concept.

\section{QR-code scanner requirments analyse}%
\label{sec:scannerRequirment}

Bij de verbetering voor de scanbaarheid op QR-codes heeft het bestuur van het bedrijf gevraagd om de manier van scannen sneller te maken. Vooreerst moet aangetoond worden op welke manier de huurders hun QR-code sneller kunnen laten scannen. Bij het scannen van QR-codes met ingebouwde USB camera’s is er theoretisch gezien geen probleem. Lockit Rentals heeft aangegeven dat door onveranderbare factoren er problematische en foutieve gevolgen zijn \autocite{Girisha2022}. Deze onveranderlijke factoren zijn vaak regen, wind, schaduw en lichtinval \autocite{Ou2014}.  Als gevolg dat de camera het beeld niet goed opneemt en hierdoor blijft de toegang geweigerd ten opzichte van de huurder. 
Bewegende QR-codes zijn moeilijk detecteerbaar, hierdoor is de festivalganger verplicht de QR-code in een stille houding te plaatsen. De QR-code zal door de festivalganger op een vaste afstand gehouden moeten worden om zo het algoritme zijn werk te laten doen\autocite{Suriyon2021}. De snelheid en de manier om een QR-code te scannen zal onder de loep genomen worden in bijhorende Proof-Of-Concept.

\section{QR-code scanner uitwerking}%
\label{sec:scannerUitwerking}

Om een snellere en efficiëntere QR-code scanner te bereiken, start het onderzoek met het analyseren van hedendaagse code. Door de scan functionaliteit onder de loep te nemen kunnen de sterktes en zwaktes geïdentificeerd worden. Hoe deze code opgebouwd is, verwijs ik door naar de sectie \ref{sec:WerkingUnits}.
Om de handelingen en ervaringen van festivalgangers zo goed mogelijk na te bootsen gaan er simulaties volgen. Deze simulaties bestaan uit test scenario’s en manieren om een QR-code te scannen. De originele code moet aangepast worden zodat we de informatie verkrijgen hoe snel het duurt om één QR-code in te lezen. Uit deze simulaties komen referentiegegevens, zoals tijdstip van uitlezen, welke data de QR-code bevat en de duur van het uitlezen van x aantal QR-codes. Om de ervaringen van de festivalgangers zo goed mogelijk na te bootsen zullen we deze experimenten uitvoeren in verschillende situaties en omgevingen.

De factoren en omgevingen die veranderen doorheen het experiment:

\begin{itemize}
    \item Normale omstandigheden
    \item Lichtinval
    \item Donkere omgeving
    \item Beschadigde QR-code     
\end{itemize}

Graag zou ik willen verwijzen naar de verkregen resultaten van het experiment waarbij de originele werkwijze voor het uitlezen van QR-codes is toegepast \ref{sec:resultatenQR-codeScanners}. Deze resultaten bieden waardevolle inzichten in het functioneren van de oorspronkelijke methode en vormen een belangrijke referentie voor het vergelijken van de prestaties van nieuwe QR-code scanner.

Het zoeken naar een betere en snellere QR-code scanner software of hardware zal de snelheid van het scannen verhogen. Om een betrouwbare vergelijking te maken tussen de originele werkwijze en de nieuwe QR-code scanner, gaan de experimenten opnieuw uitgevoerd worden. Hierdoor kunnen we de resultaten van beide methoden op een nauwkeurige manier vergelijken en kan er tot een weloverwogen conclusie gekomen worden. In het hoofdstuk dat zich richt op de verwerking van de resultaten, zal dan ook uitvoerig worden ingegaan op de verschillen en overeenkomsten tussen de twee manieren van scannen bij een locker unit.
\newpage
\section{Chatbot requirements analyse}%
\label{sec:chatbotRequirment}

Het ontwikkelen en integreren van een chatbot in een mobiele applicatie is een complex proces. Er zijn talrijke opties om een chatbot te ontwikkelen, echter moet er rekening gehouden worden met het effectieve gebruik. Om de doeleinden goed in kaart te brengen is er aan de hand van alle doeleinden een requirements analyse opgesteld samen met het bestuur van Lockit Rentals. Hierbij zijn alle opties goed afgewogen en de doeleinden opgesomd.

De chatbot moet in staat zijn om mensen te helpen met allerlei vragen. Dit kan aan de hand van slimme vragen of eventueel van vooraf gestelde vragen waarbij de gebruiker gewoon de vraag moet aanklikken. De chatbot moet telkens op een correcte manier afgesloten worden. Alles moet in een logische volgorde verlopen met bij elke interactie een gepast antwoord. Het hoofddoel van de chatbot is het teruggeven van een verloren QR-code. Op welke communicatieve manier deze persoon zijn verloren QR-code terug zal krijgen, was nog onduidelijk. Voor deze Proof-Of-Concept zullen we gebruik maken van email communicatie en niet via SMS. Op basis van eigen ondervindingen wordt geconcludeerd dat de kosten van SMS in vergelijking met e-mail aanzienlijk hoger zijn \autocite{Paavola2023} .Dit gegeven speelt een belangrijke factor bij het ontwerpen van de Lockit Rental chatbot. 

\section{Chatbot uitwerking}%
\label{sec:chatbotUitwerking}

Voor het maken van een chatbot die festivalgangers helpt bij het terugwinnen van verloren QR-code maken we gebruik van een online platform. Bij het maken van deze keuze heb ik rekening gehouden met verschillende factoren, zoals de functionaliteit, de integratiemogelijkheden, de prijs en beschikbare documentatie. De keuze van het platform moet aansluiten bij de specifieke behoeften en vereisten van Lockit Rentals. Deze behoeften zijn opgesomd in de requirements analyse \ref{sec:chatbotRequirment}.

Het is namelijk mogelijk om een chatbot zonder service te ontwikkelen maar toch heb ik gekozen om met een online service te werken. Dit biedt ons voordelen zoals technische ondersteuning, makkelijke integratie met andere systemen en snellere ontwikkelingstijd \autocite{Mercir2022} \autocite{folstad2021investigating}. De technologie van een chatbot zonder service leren en beheersen inclusief het maken van een geïntegreerd systeem dat operationeel moet zijn tegen en bepaalde einddatum is onhaalbaar. Hiervoor heb ik onderzoek gedaan naar een online chatbot platvorm die een combinatie weergeeft van beide. Hierdoor kunnen we een eenvoudige chatbot maken en integreren in een nieuw systeem in een heel korte tijdspanne. Echter kunnen we bij specifiekere gebruiksscenario’s de bot uitbreiden op vlak van functionaliteit. Het platform laat ons toe om eigen gespecialiseerde code te schrijven en uit te voeren doorheen het chatproces van de festivalganger. 

Na het maken van deze keuze, ben ik aan de slag gegaan met het uitwerken van de chatbot. Bij ontwikkeling hebben we ons gericht op een duidelijke gebruikersinterface die makkelijk gehanteerd kan worden door de eindgebruiker. Hoe deze chatbot is opgebouwd kan u vinden in \ref{sec:OntwikkelingChatbot}. Aangezien gewerkt wordt met een online service waarvan de integratiemogelijkheden goed opgebouwd zijn, is de live chatbot al in de Lockit Rentals applicatie geïntegreerd. Later kan er ook nog geopteerd worden om deze chatbot te integreren in hun eigen Facebookpagina of andere sociale media platformen. 
Tot slot verwijs ik graag naar het resultaat van de chatbot \ref{sec:resultatenChatbot}. De nieuwe chatbot kan nu succesvol QR-codes recupereren en de gevraagde informatie weergeven.





